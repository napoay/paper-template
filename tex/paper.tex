%-*- coding: UTF-8 -*-
% paper.tex
\documentclass[twocolumn]{ctexart}
\usepackage{lipsum,mwe,cuted}
\usepackage{float}%%%%提供浮动体的[H]选项,进而取消浮动
\usepackage{caption}%%提供\captionof命令
\usepackage{geometry}

\pagestyle{plain}
% \geometry{a4paper,left=2cm,right=2cm,top=2cm,bottom=2cm}

\geometry{a4paper,scale=0.8}

\title{杂谈勾股定理}
\author{ 张三\textsuperscript{1},\ 李四\textsuperscript{1}  \\\textsuperscript{1} 中国科学院大学 \ 北京 \ 100049 }
\date{\vspace{-2em}}

\stripsep 8pt
\pagestyle{plain}
\newtheorem{thm}{定理}
\bibliographystyle{unsrt}
\CTEXsetup[format={\Large\bfseries}]{section}
\newcommand{\upcite}[1]{\textsuperscript{\textsuperscript{\cite{#1}}}}


\begin{document}

\maketitle

\begin{strip}

\noindent  \textbf{摘要} \quad 这是一篇关于勾股定理的小短文。这是一篇关于勾股定理的小短文。这是一篇关于勾股定理的小短文。这是一篇关于勾股定理的小短文。这是一篇关于勾股定理的小短文。这是一篇关于勾股定理的小短文。这是一篇关于勾股定理的小短文。这是一篇关于勾股定理的小短文。

\noindent  \textbf{关键词 \quad  数学, DGA, 勾股定理, 算术}
\\
\end{strip}


\section*{1  引言}

西方称勾股定理为毕达哥拉斯定理\upcite{2dPose},将勾股定理的发现归功于公元前6世纪的毕达哥拉斯学派[1]。该学派得到了一个法则,可以求出可排成直角三角形三边的三元数组。毕达哥拉斯学派没有书面著作\upcite{antonakakis2012throw},该定理的严格表述和证明则见于欧几里\footnote{欧几里德,约公元前 330--275 年。}《几何原本》的命题 47:“直角三角形斜边上的正方形等于两直角边上的两个正方形之和。”证明是用面 积做的\upcite{RFPose}。

我国《周髀算经》载商高(约公元前 12 世纪) 答周公问:
\begin{quote}
    \zihao{-5}\kaishu 勾广三,股修四,径隅五。
\end{quote}

又载陈子(约公元前 7–6 世纪)答荣方问\upcite{kuhrer2014paint}: 
\begin{quote}
    \zihao{-5}\kaishu  若求邪至日者,以日下为勾,日高为股,勾股各自乘,并而开方除之,得邪至日。
\end{quote}
都较古希腊更早。后者已经明确道出勾股定理的一 般形式。图 1 是我国古代对勾股定理的一种证明。

满足式 (1) 的整数称为勾股数。第1节所说毕 达哥拉斯学派得到的三元数组就是\emph{勾股数}。下表列出一些较小的勾股数:都较古希腊更早。后者已经明确道出勾股定理的一 般形式。图 1 是我国古代对勾股定理的一种证明 [2]都较古希腊更早。后者已经明确道出勾股定理的一 般形式。图 1 是我国古代对勾股定理的一种证明 [2]

\section*{2 相关工作}

勾股定理可以用现代语言表述如下:


\begin{thm}[勾股定理] 

    直角三角形斜边的平方等于两腰的平方和。
    可以用符号语言表述为:设直角三角形 ABC,其中 \angle C = $90^\circ$,则有
\begin{equation}
    AB^2 = BC^2 + AC^2  
\end{equation}

\end{thm}
满足式 (1) 的整数称为勾股数。第1节所说毕 达哥拉斯学派得到的三元数组就是\emph{勾股数}。下表列出一些较小的勾股数:都较古希腊更早。后者已经明确道出勾股定理的一 般形式。图 1 是我国古代对勾股定理的一种证明 [2]都较古希腊更早。后者已经明确道出勾股定理的一 般形式。图 1 是我国古代对勾股定理的一种证明 [2]。

满足式 (1) 的整数称为勾股数。第1节所说毕 达哥拉斯学派得到的三元数组就是\emph{勾股数}。下表列出一些较小的勾股数:都较古希腊更早。后者已经明确道出勾股定理的一 般形式。图 1 是我国古代对勾股定理的一种证明 [2]都较古希腊更早。后者已经明确道出勾股定理的一 般形式。图 1 是我国古代对勾股定理的一种证明 [2]

满足式 (1) 的整数称为勾股数。第1节所说毕 达哥拉斯学派得到的三元数组就是\emph{勾股数}。下表列出一些较小的勾股数:都较古希腊更早。后者已经明确道出勾股定理的一 般形式。图 1 是我国古代对勾股定理的一种证明 [2]都较古希腊更早。后者已经明确道出勾股定理的一 般形式。图 1 是我国古代对勾股定理的一种证明 [2]

满足式 (1) 的整数称为勾股数。第1节所说毕 达哥拉斯学派得到的三元数组就是\emph{勾股数}。下表列出一些较小的勾股数:都较古希腊更早。后者已经明确道出勾股定理的一 般形式。图 1 是我国古代对勾股定理的一种证明 [2]都较古希腊更早。后者已经明确道出勾股定理的一 般形式。图 1 是我国古代对勾股定理的一种证明 [2]

\section*{3 模型和实验}

\lipsum[2-3]


\begin{strip}
    \centering\includegraphics[width=0.4\textwidth]{example-image}
    \captionof{figure}{跨栏插图成功}
\end{strip}

\lipsum[4-7]
\section*{4 总结}
\lipsum[8]
\bibliography{ref}

\end{document}